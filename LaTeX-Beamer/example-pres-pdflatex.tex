%
% THE BEER-WARE LICENSE (Rev. 42):
% Ronny Bergmann <bergmann@math.uni-luebeck.de> wrote this file. As long as you
% retain this notice you can do whatever you want with this stuff. If we meet
% some day, and you think this stuff is worth it, you can buy me a beer or
% coffee in return.
%
% This file is just to get started - You need the corresponding Logo
%
\documentclass[german,10pt,xcolor=colortbl,compress
%,draft
]{beamer}
\usepackage[utf8]{inputenc}
\usepackage[OT1]{fontenc}
\usepackage{calc}
\usepackage[ngerman]{babel} % Neue Rechtschreibung
\usepackage{amsmath,amsthm,amssymb,euscript} % AMS-LaTeX  
\usepackage{enumerate,graphicx}
% Load Theme
\usetheme[noptsans,navigation=true, FB=Mathematik, frametotal=true]{TUKL}
%
\setbeamertemplate{navigation symbols}{}
\title{Beispielpräsentation}
\subtitle{Untertitel}
\date[]{\today\\[1ex] WorkShop/Conference}
\author[Autor in Fußzeile]{Autor: Name, Vorname}
\institute[]{AG xy\\FB ab\\TU Kaiserslautern}
%Setze ein Logo auf der Titelseite unten rechts
\renewcommand{\theSecondLogo}{}

\begin{document}
	\maketitle	
	\section{Einleitung}
	\begin{frame}{Inhalt}
		\tableofcontents
	\end{frame}
	\begin{frame}{Es geht los}
		Hier in den Farben des FB-Mathematik
	\end{frame}
	\section{Erstes Thema}
	\begin{frame}{Ein erstes Thema}{Ein Untertitel}
	ABCx
	\begin{lemma}[Ein Beispiellemma]
		Ist das hier und es gilt.
	\end{lemma}
	\begin{example}
		Ein Beispiel, wie dieses hier
	\end{example}
	\alert{ACHTUNG!}
		Etwas Hervorgehobenes. Die Farben sind bisher alle dem Handbuch zum Corporate Design entnommen.
	\end{frame}
	\subsection{B}
	\begin{frame}{Ein Beweis}{Mit Untertitel}
			\begin{proof}
				Weil.
			\end{proof}
	\end{frame}
	\begin{frame}[plain]{}{}%Ohne Titel und Untertitel damits ganz leer und weiß ist
		Ich bin so ein leerer Frame
	\end{frame}
\end{document}